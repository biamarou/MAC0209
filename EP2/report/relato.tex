\documentclass[a4paper, 12pt]{article}

\usepackage[brazilian]{babel}
\usepackage[utf8]{inputenc}
\usepackage[T1]{fontenc}
\usepackage[a4paper]{geometry}
\usepackage{amsmath}
\usepackage{amssymb}
\usepackage{indentfirst}
\usepackage{graphicx}
\usepackage[colorlinks=true, linkcolor=red]{hyperref}
\renewcommand{\rmdefault}{ptm}

\title{EP2 Relato}
\author{Beatriz F. Marouelli, Bruno B. Scholl, \\João H. Luciano, Leonardo Lana Violin Oliveira, \\ Lucas B. Yau, Victor H. Seiji}
\date{8 de Maio de 2017}

\begin{document}
\maketitle

\section*{Movimentos escolhidos}
Pêndulo simples e movimento circular uniforme.

\section*{Protocolos de aquisição}

\subsection*{Pêndulo}
Amarramos o celular com barbante, e destes barbantes fizemos dois eixos para fixar o celular no cano de forma mais estável, desta forma o fio que medimos é a reta entre o cano e o celular. Deixamos o pêndulo encostado na altura do peito do Leonardo por 5 segundos para estabilização e paramos o pêndulo após dez balançadas, paramos o pêndulo no eixo central por 5 segundos.

\subsection*{Movimento circular uniforme}
Colocamos um celular em cada pá do ventilador de teto para balancear o peso. Esperamos 5 segundos, para termos um parâmetro de estabilização. Depois temos 10 segundos de aceleração do ventilador, e 5 segundos de movimento circular uniforme, depois disso paramos o celular de novo para termos outra estabilização.

\section*{Métodos}
Para as simulações usamos os algoritmos de Euler-Cromer e Euler-Richardson.

\section*{Progresso}
Até agora coletamos os dados do pêndulo simples. E percebemos que o \textit{Physics Tool Box} corta 10 segundos de regitros que passam de 30 segundos, isto é, qualquer registro entre 30 e 40 segundos é perdido.


\end{document}
