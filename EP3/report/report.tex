\documentclass[a4paper, 12pt]{article}

\usepackage[brazilian]{babel}
\usepackage[utf8]{inputenc}
\usepackage[T1]{fontenc}
\usepackage[a4paper]{geometry}
\usepackage{amsmath}
\usepackage{amssymb}
\usepackage{indentfirst}
\usepackage{graphicx}
\usepackage{float}
\usepackage[colorlinks=true, linkcolor=blue]{hyperref}
\renewcommand{\rmdefault}{ptm}

\title{Relatório EP2}
\author{Beatriz F. Marouelli, Bruno B. Scholl, \\João H. Luciano, Leonardo Lana Violin Oliveira, \\ Lucas B. Yau, Victor H. Seiji}
\date{03 de Julho de 2017}

\begin{document}
\maketitle

\section*{Introdução}
Para este exercício-programa fizemos os itens do Exercício 14.5 do livro
\textit{An Introduction to Computer Simulation Methods Aplications to Physical
Systems}.

\section*{Avisos}
Para realizar os exercícios abaixo foram modificados os programas em java fornecidos
pelo pacote do livro. Alguns dos parâmetros não podiam ser modificados, como os eixos
dos gráficos, mas contextualizando em cada item é possível inferir qual seria a marcação
correta.

Em relação aos vídeos, gravamos as simulações com o seguinte padrão: começar a partir do
step 100 (recomendação do livro de evoluir o sistema antes de iniciar a gravação), e parar
no step 300.

\section*{Respostas dos itens}
\subsection*{A}
Conforme a densidade aumenta (aumento do número de carros maior que o aumento do
tamanho da pista) o número de congestionamentos (números de grandes blocos verdes 
na simulação) aumenta também. A partir da densidade de $0.140$ há trânsito, como 
descobrimos experimentalmente através das simulações.

Como podemos observar nos gráficos gerados de \textit{flow X density},
inicialmente temos um fluxo baixo, devido à baixa densidade, seguido de
um pico no fluxo quando a densidade atinge cerca de $0.2$. Após este pico,
o fluxo inicia um decaimento conforme a densidade aumenta.  
\subsection*{B}

\subsection*{C}
Os métodos foram adicionados à classe \textit{FreeWay}. Os gráficos gerados 
por esses métodos podem ser observados nos vídeos do item d).

\subsection*{D}
Quando o parâmetro velocidade foi modificado para $v_{max} = 1$, o fluxo de carros é
bem uniforme, sem o surgimento de aglomerados. No caso em que $v_{max} = 2$, começam
a surgir pequenos agrupamentos, mas que se desfazem rapidamente. Por fim,
quando a velocidade é alterada para $v_{max} = 5$, congestionamentos mais demorados
aparecem. 

Isto se deve porque quanto maior a velocidade, mais rapidamente os carros 
percorrem a via, e por consequência encontram com mais rapidez um carro
que tenha sido aleatóriamente escolhido para frear, gerando o engarrafamento.

\subsection*{E}
Neste item modificamos a probabilidade de um carro ser escolhido aleatoriamente para ter sua
velocidade reduzida. Primeiro ela é modificada para $p = 0.2$, e podemos observar nos
vídeos que não ocorrem aglomerados, pois pouquíssimos carros tem sua velocidade reduzida.
Já quando modificamos a probabilidade para $p = 0.8$, fica bem visível a formação de
congestionamentos, pois a chance de um carro ter sua velocidade reduzida é alta, obrigando muitos
outros carros a abaixarem sua velocidade também. 

\subsection*{F}
Este item foi simulado com 10 carros e 20 carros em circulação. Quando apenas 10 carros estão
circulando não há grande variação no fluxo em relação à simulação sem rampas de entrada e saída.
No entanto, ao aumentar o número de carros, grandes blocos de aglomeração começam a surgir. Isto
ocorre porque quando um carro entra na via principal, sua velocidade está menor em relação aos que
já se encontram nela, forçando esses carros a reduzirem de velocidade. O mesmo ocorre na saída
do carro da via principal, ele reduz de velocidade primeiro, o que também força a redução da velocidade
dos carros que estão atrás dele.

\subsection*{G}
Neste item também utilizamos os parâmetros de 10 e depois 20 carros em circulação.
Novamente, quando temos apenas 10 carros em circulação não é notada grande mudança de fluxo 
em relação à situação de uma via simples, apenas alguns pequenos agrupamentos por consequência
de uma mudança de faixa, mas que se dissipam rapidamente.
Quando modificamos o número de carros para 20, já ficam mais visíveis os congestionamentos causados
pela mudança de faixa dos carros, que forçam a redução da velocidade dos demais.

\subsection*{H}


\section*{Contribuições}
As contribuições dos membros do grupo em ordem alfabética:

\textbf{Beatriz}: Gravações das animações, itens D e E e relatório.

\textbf{Bruno}: Edição do vídeo e item A.

\textbf{João}: Relatório.

\textbf{Leonardo}: Relatório e itens A e B.

\textbf{Lucas}: item F.

\textbf{Victor}: itens C, G e H.

\section*{Link do vídeo}
Para assistir o vídeo, clique
\href{https://www.youtube.com/watch?v=Tx7dhtJjCZc&feature=youtu.be}{aqui}

\end{document}
